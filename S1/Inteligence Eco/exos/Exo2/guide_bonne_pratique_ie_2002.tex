 \part{"Bonne pratique en matière d'Intelligence Économique..." 2002}
 
	 
 
 
 Le titre de ce texte est "Bonne pratique en matière d'Intelligence Économique..." et il fut publié en 2002 par l'éditeur \emph{Guide Pratique et Découverte}.
 Le texte traite d'intelligence économique .Dans la forme le texte est un roman avec une narration à la première personne . Le narrateur est le personnage principale . Dans ce livre on part à la découverte de l'intelligence économique. Pas à pas , le narrateur ajoute de nouvelles notions qui composent l'intelligence économique. Ainsi le héros découvre les tenants et aboutissant de l'intelligence au fur et à mesure qu'il visite les entreprises .

Lors de sa première visite le héros passe un entretien avec une entreprise dans la proximité de Saint-Dié . Le but de cette visite est de : "discuter de la collecte de l'information "\footnote{p.7}. Dés lors, le narrateur explique que pour lui l'\ie est une matière floue et il pense que ses méthodes sont efficaces. Or, très vite lors de la discussion il apprend que l'on distingue deux types d'informations : l'information technique et technologique, et l'information de veille stratégique. De plus, lors du deuxième volet, celui traitant de la veille stratégique , on comprend que chaque cible est identifié et que pour chaque cible un outils de collecte d'information est mis en place . Enfin le narrateur nous offre un bilan de son entretien :
\center{
	\emph{
		"La créativité ne peut s'exprimer que si l'on dispose de l'information et qu'on la maitrise ..."
	}\footnote{cf. p.9}
}
Lors de sa deuxième rencontre, l'interlocuteur est un chef d'entreprise évoluant dans l'industrie de la chimie.Plus tard il est préciser que l'entreprise est principalement orienté vers l'export. Ainsi lors de cette entretien, le narrateur apprends qu'il faut collecter ces informations mais se laisser emporter par le sont flot. Mais aussi qu'en "analysant et en tirant parti d'informations pertinentes "\footnote{cf. p.15} on peut espérer en profiter .

Lors de son troisième entretien, le narrateur va rencontrer une entreprise avec laquelle il va s'entretenir de la diffusion en interne de l'information et de sa mémorisation.Finalement, il comprend que "la diffusion et le partage de l'information ne se décrètent, ni ne s'imposent"\footnote{cf. p.21}. Ainsi, il ajoute qu'"ils sont l'aboutissement à la fois d'une démarche personnelle et de la mise en place de structures d'échanges"\footnote{cf. p.21}. Et il finit par comprendre l'importance de la mémorisation de l'information.

Lors de son quatrième entretien, le héros à rendez-vous dans une usine . La leçon tiré dans lors de cette rencontre l'importance de savoir ce protéger et de savoir protéger les informations de son entreprise.

Lors de la dernière rencontre , le narrateur se déplace dans une entreprise de fabrication de mono-cristaux. Il comprend finalement que rien ne sert de se presser : "rien ne sert de courir, il faut partir à point"\footnote{cf. p.34}
 