\part{Comparatif}

	\section{La forme} 
	Le guide de 2002 "Bonne pratique en matière d'Intelligence Économique..." prends la forme d'un roman raconté à la première personne, où le narrateur va d'une entreprise à autre chapitre après chapitre . Le ton y est informel, le narrateur n'hésitant pas faire de l'humour. 
	\\Le guide de 2009 "Guide des bonnes pratiques en matière d'intelligence économique" prend, quant à lui, prend la forme d'un rapport structuré et formel .
	\\Finalement, le dernier guide de 2012 "une démarche d'intelligence économique dans mon entreprise ?" prend la forme d'une fiche informative. On y trouve une présentation proche de l'affiche publicitaire, ceci est visible de part: l'utilisation de la police d'écriture, la présence de beaucoup de couleur, la présence de slogan quasi publicitaire.  	
	
	\section{Le fond}
	Le guide de 2009 est le plus complet lorsqu'il s'agit du fond. Il est structuré, contient aussi bien de l'infographie, que des exemples, mais aussi des définitions. De plus, ce guide traite de tout les aspects de l'\ie .
	\\le guide de 2002 est dans le fond une démonstration par l'exemple. C'est à dire que le narrateur au fur et à mesure de ses rendez-vous introduit les différents aspect de l'\ie et brise les idées reçues sur l'\ie.
	\\le guide de 2012 est moins fournie sur le fond mais touche de par sa forme simple et publicitaire .