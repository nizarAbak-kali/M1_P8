 \part{"Guide des bonnes pratiques en matière d'intelligence économique" 2009}
 
 Il s'agit d'un guide de bonne pratique en matière d'intelligence économique diffusé par le ministère français  de l'économie, de l'industrie et de l'emploi, le 9 mars 2009 . Cet ouvrage à pour but de guider les entreprises à intégrer une démarche d'intelligence économique dans la politique de l'entreprise. Ce texte prend la forme d'un rapport structuré, où, chaque partie explique les avantages et la démarche de la mise en place d'un service d'intelligence économique .  \\
 
 
 Dans un premier temps, le guide explique quelles sont les raisons de mettre en œuvre une démarche d'\ie . Ainsi on comprend que la mise en place de ce système est due à plusieurs facteur tels que la mondialisation, le développement rapide des nouvelles technologies de l'information. Puis on explique l'organisation de l'\ie . "On peut distinguer trois modèles d'organisation de l'intelligence économique en entreprise." \footnote{cf. p.9}.Le texte présente 3 types d'organisation :
 \begin{itemize}
 	\item[•] "L'intelligence économique est confiée à un responsable spécialisé au sein de l'entreprise"\footnote{cf. p.9}.
 	\item[•]" L'intelligence économique est confiée à une personne ayant d'autres responsabilités au sein de l'entreprise"\footnote{cf. p.9}.
	\item[•] "Le schéma type dans les PME\footnote{Petites Moyennes Entreprises}"\footnote{cf. p.9}.
 \end{itemize}
 Dans un second temps, le guide nous expose les enjeux stratégique en rapport avec l'\ie .Ainsi on comprend que la collecte d'information n'est pas l'enjeu principale mais plutôt la compréhension des besoins de l'entreprise et la définition exact des information que l'on veut collecter .\\
 La troisième partie de ce guide traite de la maitrise de l'information. C'est à dire , non seulement la collecte intelligente  d'information répondant aux besoins de l'entreprise mais aussi de la manière d'exploiter ces informations afin de les rendre profitable à l'entreprise. \\
 La partie suivante explique le processus de valorisation dans l'entreprise mais aussi à l'extérieur.Lorsqu'il s'agit de valorisation de l'information dans l'entreprise l'enjeu est de la mettre disposition , suite à une validation préalable, puis de la sécuriser.\\

Dans la cinquième partie, le guide présente les moyens de protection des informations de l'entreprise.C'est à dire , la mise en place de dispositif de sensibilisation et de formation du personnel face aux risques due au fuite d'information sensible de l'entreprise.Puis, le guide traite des systèmes de protections qu'une entreprise doit mettre en place : des règles, des moyens organisationnels... \\

 Dans la partie suivante on traite de l'aspect juridique de l'\ie .Cette partie parle, de protection juridique aussi bien au niveau national(lors d'une collecte d'information), que international lors "diffusion d'informations sensibles dans le cadre de procédures avec des autorités publiques étrangères"\footnote{cf. p.39}


%%%%%%%%%%%%%%%%%%%%%%%%%%%%%%%%%%%%%%%%%%%%%%%%%%%%%%%%%%%%%%%%%%%%%%%%%%%%%%%%%%%%%%%%%%%%

